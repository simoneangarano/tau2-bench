\section{Additional Experimental Results}
\label{app:results}

\subsection{Voice Interaction Quality: Full Metric Breakdown}
\label{app:voice-metrics-detail}

Table~\ref{tab:voice-quality-detail} provides the full breakdown of voice interaction metrics. Columns are grouped by: \textbf{Latency} ($L_R$ = Response Latency, $L_Y$ = Yield Latency), \textbf{Responsiveness} ($R_R$ = Response Rate, $R_Y$ = Yield Rate), \textbf{Interrupt} ($I_A$ = Agent Interruption Rate), and \textbf{Selectivity} ($S_{BC}$ = Backchannel Correct, $S_{VT}$ = Vocal Tic Correct, $S_{ND}$ = Non-Directed Correct). For $L_R$, $R_R$, and $I_A$, separate columns show Clean (C) and Realistic (R) speech conditions; other metrics are evaluated on Realistic only.

\begin{table*}[h]
\caption{Voice interaction quality metrics---full breakdown (Realistic condition). \textbf{Bold} indicates best per domain. $\uparrow$ = higher is better, $\downarrow$ = lower is better.}
\label{tab:voice-quality-detail}
\centering
\begin{small}
\resizebox{\textwidth}{!}{%
\begin{tabular}{@{}ll|cc|c|cc|c|cc|ccc@{}}
\toprule
 & & \multicolumn{3}{c|}{\textbf{Latency}$\downarrow$} & \multicolumn{3}{c|}{\textbf{Responsiveness}$\uparrow$} & \multicolumn{2}{c|}{\textbf{Interrupt}$\downarrow$} & \multicolumn{3}{c}{\textbf{Selectivity}$\uparrow$} \\
 & & \multicolumn{2}{c|}{$L_R$} & \multirow{2}{*}{$L_Y$} & \multicolumn{2}{c|}{$R_R$} & \multirow{2}{*}{$R_Y$} & \multicolumn{2}{c|}{$I_A$} & \multirow{2}{*}{$S_{BC}$} & \multirow{2}{*}{$S_{VT}$} & \multirow{2}{*}{$S_{ND}$} \\
\textbf{Domain} & \textbf{Provider} & C & R & & C & R & & C & R & & & \\
\midrule
\multirow{3}{*}{All} & Google & 1.36s & 1.44s & \textbf{0.82s} & \textbf{98\%} & 86\% & 57\% & \textbf{9\%} & \textbf{24\%} & 90\% & 34\% & 30\% \\
 & OpenAI & 3.65s & 3.24s & 1.20s & 89\% & 77\% & 59\% & 40\% & 34\% & \textbf{95\%} & \textbf{71\%} & \textbf{57\%} \\
 & xAI & \textbf{0.90s} & \textbf{0.99s} & 1.00s & 95\% & \textbf{90\%} & \textbf{79\%} & 49\% & 104\% & 81\% & 53\% & 20\% \\
\midrule
\multirow{3}{*}{Retail} & Google & 1.43s & 1.52s & \textbf{0.78s} & \textbf{97\%} & 85\% & 59\% & \textbf{9\%} & \textbf{28\%} & 80\% & 35\% & 37\% \\
 & OpenAI & 3.69s & 2.94s & 1.06s & 83\% & 78\% & 65\% & 36\% & 30\% & \textbf{92\%} & \textbf{67\%} & \textbf{53\%} \\
 & xAI & \textbf{0.93s} & \textbf{0.93s} & 1.02s & 91\% & \textbf{89\%} & \textbf{73\%} & 72\% & 118\% & 82\% & 64\% & 21\% \\
\midrule
\multirow{3}{*}{Airline} & Google & 1.38s & 1.43s & \textbf{0.81s} & \textbf{98\%} & 87\% & 57\% & \textbf{8\%} & \textbf{21\%} & 90\% & 31\% & 28\% \\
 & OpenAI & 3.64s & 3.39s & 1.32s & 88\% & 77\% & 56\% & 48\% & 39\% & \textbf{100\%} & \textbf{70\%} & \textbf{61\%} \\
 & xAI & \textbf{0.90s} & \textbf{1.07s} & 0.96s & 95\% & \textbf{88\%} & \textbf{83\%} & 39\% & 110\% & 71\% & 45\% & 27\% \\
\midrule
\multirow{3}{*}{Telecom} & Google & 1.28s & 1.36s & \textbf{0.87s} & \textbf{99\%} & 86\% & 53\% & \textbf{10\%} & \textbf{23\%} & \textbf{100\%} & 35\% & 25\% \\
 & OpenAI & 3.62s & 3.38s & 1.21s & 95\% & 77\% & 57\% & 37\% & 32\% & 94\% & \textbf{77\%} & \textbf{58\%} \\
 & xAI & \textbf{0.86s} & \textbf{0.97s} & 1.01s & 99\% & \textbf{93\%} & \textbf{81\%} & 34\% & 85\% & 90\% & 50\% & 13\% \\
\bottomrule
\end{tabular}
}
\end{small}
\end{table*}

% TODO: Add per-task breakdown, statistical significance tests

\subsection{Qualitative Error Analysis}
\label{app:qualitative-analysis}

We conducted a qualitative analysis of task failures to understand error sources and types. We sampled 20 failed tasks from two analysis cohorts: (1) Voice-Fragile (tasks passing in text but failing in Clean audio), and (2) Noise-Fragile (tasks passing in Clean but failing in Realistic audio).

\paragraph{Qualitative Annotations.}
Table~\ref{tab:full-notes-both} shows the qualitative annotations for each sampled failure.

\begin{table*}[h]
\caption{Qualitative error annotations for sampled task completion failures. Left: Voice-Fragile cohort (failures from Clean audio setting, 20 tasks). Right: Noise-Fragile cohort (failures from Realistic audio setting, 20 tasks).}
\label{tab:full-notes-both}
\centering
\begin{small}
\begin{minipage}{0.48\textwidth}
\centering
\textbf{Voice-Fragile}\\[0.5em]
\begin{tabular}{clll}
\toprule
\textbf{Task} & \textbf{Provider} & \textbf{Source} & \textbf{Type} \\
\midrule
8 & Google & agent & logical \\
9 & xAI & agent & vad \\
16 & xAI & agent & logical \\
17 & xAI & agent & logical \\
26 & Google & agent & transcription \\
30 & OpenAI & agent & transcription \\
31 & xAI & agent & vad \\
37 & xAI & agent & logical \\
49 & xAI & agent & transcription \\
51 & xAI & agent & transcription \\
54 & OpenAI & user & early\_termination \\
58 & OpenAI & user & hallucination \\
59 & OpenAI & agent & logical \\
78 & OpenAI & agent & unresponsive \\
85 & Google & user & hallucination \\
86 & Google & user & hallucination \\
87 & OpenAI & agent & logical \\
98 & Google & user & logical \\
104 & Google & agent & logical \\
106 & Google & agent & logical \\
\bottomrule
\end{tabular}
\end{minipage}
\hfill
\begin{minipage}{0.48\textwidth}
\centering
\textbf{Noise-Fragile}\\[0.5em]
\begin{tabular}{clll}
\toprule
\textbf{Task} & \textbf{Provider} & \textbf{Source} & \textbf{Type} \\
\midrule
1 & xAI & agent & unresponsive \\
5 & OpenAI & agent & unresponsive \\
14 & Google & agent & logical \\
18 & Google & user & hallucination \\
23 & xAI & agent & logical \\
24 & OpenAI & agent & vad \\
39 & OpenAI & agent & logical \\
44 & Google & agent & transcription \\
46 & xAI & agent & transcription \\
53 & OpenAI & agent & logical \\
56 & OpenAI & agent & logical \\
62 & xAI & user & logical \\
68 & xAI & agent & transcription \\
73 & Google & agent & vad \\
75 & Google & agent & transcription \\
77 & xAI & agent & unresponsive \\
80 & OpenAI & agent & unresponsive \\
84 & OpenAI & agent & unresponsive \\
88 & Google & agent & logical \\
94 & xAI & agent & unresponsive \\
\bottomrule
\end{tabular}
\end{minipage}
\end{small}
\end{table*}

\paragraph{Error Type Definitions.}
We categorize errors into six types based on observed failure patterns:

\begin{itemize}[nosep,leftmargin=*]
    \item \textbf{Logical} (Agent or User): Reasoning or execution errors, including incorrect tool call arguments/formatting, taking wrong actions (cancelling/modifying wrong items), failing to follow instructions (not asking for spelling, not confirming), or losing track of conversation state.
    
    \item \textbf{Transcription} (Agent): Speech-to-text errors where the agent incorrectly transcribes user speech, most commonly during authentication when users spell names/emails letter-by-letter, or when transcribing specific user requests.
    
    \item \textbf{VAD} (Agent): Voice Activity Detection errors where the agent fails to correctly detect when the user has spoken, causing it to miss confirmations, ask redundant questions, or respond at inappropriate moments.
    
    \item \textbf{Unresponsive} (Agent): Agent goes silent for an extended period (20--30+ seconds) or fails to respond despite multiple user check-ins. We suspect many of these are caused by VAD failures on specific speech patterns (e.g., letter-by-letter spelling, short user utterances). Categorized separately because extended unresponsiveness should never occur in production systems.
    
    \item \textbf{Hallucination} (User): User simulator states information not present in the task instructions or contradicts available information, causing task failure.
    
    \item \textbf{Early Termination} (User): User ends the call prematurely before the task is fully completed, often due to ambiguous communication where user assumes the task is done when it is not.
\end{itemize}

\subsection{Statistical Reliability Analysis}
\label{app:stat-sig}

To assess statistical reliability, we conducted 3 independent runs per condition on the Retail domain (n=114 tasks per run). Table~\ref{tab:stat-sig} reports mean pass@1 with 95\% confidence intervals.

\begin{table}[h]
\caption{Statistical reliability analysis for Retail domain (3 runs, n=114 tasks each). All reported values are mean $\pm$ 95\% CI.}
\label{tab:stat-sig}
\centering
\begin{small}
\begin{tabular}{llc}
\toprule
\textbf{Condition} & \textbf{Model} & \textbf{Mean $\pm$ 95\% CI} \\
\midrule
\multirow{2}{*}{Text} & GPT-4.1 & 73.4\% $\pm$ 2.5\% \\
 & GPT-5 & 81.9\% $\pm$ 0.6\% \\
\midrule
\multirow{3}{*}{Voice (Clean)} & Google & 38.3\% $\pm$ 2.3\% \\
 & OpenAI & 39.2\% $\pm$ 2.5\% \\
 & xAI & 36.3\% $\pm$ 5.8\% \\
\midrule
\multirow{3}{*}{Voice (Realistic)} & Google & 26.0\% $\pm$ 3.2\% \\
 & OpenAI & 12.0\% $\pm$ 3.8\% \\
 & xAI & 21.1\% $\pm$ 1.7\% \\
\bottomrule
\end{tabular}


\end{small}
\end{table}

The confidence intervals confirm that both gaps are statistically significant with non-overlapping 95\% CIs: (1) the text-to-Clean gap---even the best voice provider under Clean conditions (OpenAI at 39.2\% $\pm$ 2.5\%, upper bound 41.7\%) does not overlap with the weaker text baseline (GPT-4.1 at 73.4\% $\pm$ 2.5\%, lower bound 70.9\%); and (2) the Clean-to-Realistic gap---all three providers show non-overlapping CIs between Clean and Realistic conditions.

\paragraph{Provider Comparisons.} Under Clean conditions, the three providers' CIs overlap substantially (Google [36.0\%, 40.6\%], OpenAI [36.7\%, 41.7\%], xAI [30.4\%, 42.1\%]), so we cannot distinguish provider performance with statistical significance. Under Realistic conditions, however, provider rankings are statistically significant: Google is best (26.0\%, CI [22.8\%, 29.2\%]), xAI is second (21.1\%, CI [19.3\%, 22.8\%]), and OpenAI is worst (12.0\%, CI [8.2\%, 15.7\%])---all pairwise CIs are non-overlapping.
