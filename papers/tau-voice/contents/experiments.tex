\section{Experimental Setup}
\label{sec:experiments}

\subsection{Domains and Tasks}

We evaluate on three domains from \tautwobench{}, totaling 278 tasks:
\begin{itemize}[nosep,leftmargin=*]
    \item \textbf{Retail} (114 tasks): Returns, exchanges, cancellations, and order modifications---often combined in a single conversation. Many tasks require handling ambiguous requests or customers who change their mind mid-conversation.
    \item \textbf{Airline} (50 tasks): Flight changes, cancellations, seat upgrades, and booking modifications requiring verification of passenger details and fare rules.
    \item \textbf{Telecom} (114 tasks): Plan changes, billing inquiries, service activations, and account modifications involving authentication and policy verification.
\end{itemize}
We designate \textbf{Retail as the primary evaluation domain} due to its heavy reliance on slot filling---collecting names, emails, order IDs, and addresses---where end-to-end speech systems are known to struggle~\citep{li_cb-whisper_2024, si_spokenwoz_2025}. Airline and Telecom serve as supporting domains to test generalization.

\subsection{Models}

We evaluate three audio-native providers, all released in the latter half of 2025:

\begin{table}[h]
\caption{Audio-native models evaluated.}
\label{tab:models}
\centering
\begin{small}
\resizebox{\columnwidth}{!}{%
\begin{tabular}{llll}
\toprule
\textbf{Provider} & \textbf{Model} & \textbf{Protocol} & \textbf{Release} \\
\midrule
OpenAI & \texttt{gpt-realtime-2025-08-28} & WebSocket & Aug 2025 \\
Google & \texttt{gemini-live-2.5-flash} & WebSocket & Dec 2025 \\
xAI & \texttt{grok-voice-agent} & WebSocket & Dec 2025 \\
\bottomrule
\end{tabular}%
}
\end{small}
\end{table}

All models receive identical system prompts with voice-specific guidance: when collecting names, emails, or IDs, ask customers to spell letter-by-letter; if authentication fails, explicitly request spelling again.

\subsection{Evaluation Conditions}

We evaluate each provider under two speech complexity conditions:

\begin{table}[h]
\caption{Speech complexity conditions: Clean vs Realistic.}
\label{tab:conditions}
\centering
\begin{small}
\resizebox{\columnwidth}{!}{%
\begin{tabular}{@{}llll@{}}
\toprule
\textbf{Category} & \textbf{Setting} & \textbf{Clean} & \textbf{Realistic} \\
\midrule
\multirow{1}{*}{Accents} & Personas & American & Diverse accents \\
\midrule
\multirow{5}{*}{Audio/Channel} & Background noise & None & Indoor/outdoor \\
 & Burst noise & None & $\sim$1/min \\
 & Frame drops & None & $\sim$2.0\% (G-E model) \\
 & Telephony & G.711 $\mu$-law 8kHz & G.711 $\mu$-law 8kHz \\
 & Muffling & None & Dynamic \\
\midrule
\multirow{4}{*}{User Behavior} & Involuntary sounds & None & Coughs, sneezes \\
 & Non agent-directed speech & None & ``hold on'', ``one sec'' \\
 & Interruptions & None & LLM-based \\
 & Backchanneling & None & LLM-based \\
\bottomrule
\end{tabular}%
}
\end{small}
\end{table}

\textbf{Clean} simulates an idealized telephony scenario: clear American-accented speech with no background noise or user interruptions. \textbf{Realistic} reflects realistic phone interactions: diverse speaker accents, environmental noise (indoor/outdoor backgrounds, burst sounds), channel degradation (frame drops, muffling), and natural user behaviors (interruptions, backchanneling, vocal tics, non-directed speech). To isolate the contribution of each factor, we also evaluate intermediate ablation conditions adding noise, accents, or user behaviors independently (Table~\ref{tab:conditions-ablation-single}).

\begin{table}[h]
\caption{Speech complexity conditions by ablation (single factors). Columns: Cln=Clean, +N=Noise, +A=Accents, +I=Interrupts, Real=Realistic (all effects).}
\label{tab:conditions-ablation-single}
\centering
\begin{small}
\resizebox{\columnwidth}{!}{%
\begin{tabular}{llccccc}
\toprule
\textbf{Category} & \textbf{Setting} & \textbf{Cln} & \textbf{+N} & \textbf{+A} & \textbf{+I} & \textbf{Real} \\
\midrule
\multirow{1}{*}{Accents} & Personas &  &  & \checkmark &  & \checkmark \\
\midrule
\multirow{5}{*}{Audio/Channel} & Background noise &  & \checkmark &  &  & \checkmark \\
 & Burst noise &  & \checkmark &  &  & \checkmark \\
 & Frame drops &  & \checkmark &  &  & \checkmark \\
 & Telephony & \checkmark & \checkmark & \checkmark & \checkmark & \checkmark \\
 & Muffling &  & \checkmark &  &  & \checkmark \\
\midrule
\multirow{4}{*}{User Behavior} & Involuntary sounds &  &  &  & \checkmark & \checkmark \\
 & Non agent-directed speech &  &  &  & \checkmark & \checkmark \\
 & Interruptions &  &  &  & \checkmark & \checkmark \\
 & Backchanneling &  &  &  & \checkmark & \checkmark \\
\bottomrule
\end{tabular}%
}
\end{small}
\end{table}

This 3$\times$3$\times$2 design (3 providers $\times$ 3 domains $\times$ 2 conditions) isolates the impact of acoustic realism on task completion. Ablation conditions are evaluated on the Retail domain to identify which factors contribute most to performance degradation.

\subsection{Simulation Parameters}

Each task runs with a fixed seed for reproducible effect scheduling (noise timing, frame drops), though LLM responses remain non-deterministic. Reproducibility refers to controlled inputs and deterministic non-LLM components; stochasticity arises from agent and simulator LLMs. Key parameters: tick duration 200ms, max conversation 1200s, user simulator LLM GPT-4.1, TTS via ElevenLabs v3 at 24kHz, interruption and backchannel check every 2s.

\subsection{Metrics}

\textbf{Task Completion:} Following \tautwobench{}, tasks are fully verifiable: success is deterministically evaluated by comparing the end state of the environment (e.g., database records) against a gold standard. We report pass@1---the proportion of tasks completed successfully on a single attempt.

\textbf{Voice Interaction Quality:} Beyond task completion, we evaluate how well agents manage real-time conversation. Effective turn-taking requires \textit{responsiveness} (acting when action is needed), \textit{latency} (reacting quickly), \textit{not interrupting} (good timing), and \textit{selectivity} (ignoring backchannels and non-directed speech). We measure:
\begin{itemize}[nosep,leftmargin=*]
    \item \textbf{Responsiveness:} Response Rate ($R_R$, proportion of user turns receiving a response) and Yield Rate ($R_Y$, proportion of interruptions where agent yields within 2s).
    \item \textbf{Latency:} Response Latency ($L_R$, time from user utterance end to agent response) and Yield Latency ($L_Y$, time to stop speaking after interruption).
    \item \textbf{Interrupt:} Agent Interruption Rate ($I_A$, proportion of turns where agent speaks before user finishes; $>$100\% means multiple interruptions per turn).
    \item \textbf{Selectivity:} Correctly ignoring backchannels ($S_{BC}$), vocal tics ($S_{VT}$), and non-directed speech ($S_{ND}$).
\end{itemize}
We report four aggregate scores: \textbf{Responsiveness} $= \mathrm{avg}(R_R, R_Y)$, \textbf{Latency} $= \mathrm{avg}(L_R, L_Y)$, \textbf{Interrupt} $= I_A$, and \textbf{Selectivity} $= \mathrm{avg}(S_{BC}, S_{VT}, S_{ND})$. See Appendix~\ref{app:voice-metrics} for detailed definitions.
