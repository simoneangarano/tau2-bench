\section{Results}
\label{sec:results}

\subsection{Quantitative Results}

\subsubsection{Task Completion}

Figure~\ref{fig:headline} and Table~\ref{tab:text-control-regular} present our headline finding: \textbf{voice agents show substantial drops from text baselines}. Under \textbf{Clean} conditions (studio-quality audio, American accents), the best voice provider already drops 38pp from GPT-5 (42\% vs GPT-5 at 80\%). Under \textbf{Realistic} conditions (background noise, diverse accents, natural user behaviors), performance drops an additional 12pp to 30\%. This gap persists even against non-reasoning text models: compared to GPT-4.1 (54\%), voice still drops 12pp (Clean) to 24pp (Realistic).

\begin{table}[h]
\caption{Text vs Voice comparison (pass@1). Text shows GPT-5 (reasoning) and GPT-4.1 (non-reasoning). Voice evaluated under Clean and Realistic conditions. Deltas show gap from GPT-5.}
\label{tab:text-control-regular}
\centering
\begin{small}
\resizebox{\columnwidth}{!}{%
\begin{tabular}{@{}llccc@{}}
\toprule
 &  &  & \multicolumn{2}{c}{\textbf{Voice}} \\
\cmidrule(l){4-5}
\textbf{Domain} & \textbf{Provider} & \textbf{Text} & \textbf{Clean} & \textbf{Realistic} \\
\midrule
\multirow{3}{*}{All} & Google & \multirow{3}{*}{79\% (54\%)} & 29\% (-50) & 24\% (-55) \\
 & OpenAI &  & 33\% (-46) & 19\% (-60) \\
 & xAI &  & \textbf{42\% (-37)} & \textbf{30\% (-49)} \\
\midrule
\multirow{3}{*}{Retail} & Google & \multirow{3}{*}{81\% (74\%)} & 39\% (-42) & \textbf{28\% (-53)} \\
 & OpenAI &  & 39\% (-42) & 15\% (-65) \\
 & xAI &  & \textbf{42\% (-39)} & 20\% (-61) \\
\midrule
\multirow{3}{*}{Airline} & Google & \multirow{3}{*}{62\% (56\%)} & 28\% (-34) & 26\% (-36) \\
 & OpenAI &  & \textbf{36\% (-26)} & 28\% (-34) \\
 & xAI &  & 26\% (-36) & \textbf{34\% (-28)} \\
\midrule
\multirow{3}{*}{Telecom} & Google & \multirow{3}{*}{95\% (34\%)} & 20\% (-75) & 19\% (-76) \\
 & OpenAI &  & 23\% (-72) & 14\% (-81) \\
 & xAI &  & \textbf{58\% (-37)} & \textbf{36\% (-59)} \\
\bottomrule
\multicolumn{5}{l}{\footnotesize \textit{Text column: GPT-5, reasoning (GPT-4.1, best non-reasoning model). Deltas relative to GPT-5.}} \\
\end{tabular}
}
\end{small}
\end{table}

The 12pp drop from Clean to Realistic conditions accounts for roughly one-quarter of the total voice-text gap; the remaining three-quarters reflects the drop from text to Clean voice.

Across providers, \textbf{xAI achieves the highest scores} (42\% Clean, 30\% Realistic), while \textbf{Google shows the smallest degradation} under realistic conditions ($-$5pp vs $-$12--14pp for others). Domain-specific patterns emerge: xAI substantially outperforms others in Telecom (59\% Clean vs 20--24\% for others), while performance is more similar across providers in Retail and Airline.

\paragraph{Statistical Reliability.} For Retail, where we conducted 3 independent runs per condition, both the text-to-Clean gap and the Clean-to-Realistic gap are statistically significant (non-overlapping 95\% CIs). Voice providers achieve 36--39\% $\pm$ 3--6pp (Clean) and 12--26\% $\pm$ 2--4pp (Realistic), compared to text baselines of 73\% $\pm$ 3pp (GPT-4.1) and 82\% $\pm$ 1pp (GPT-5). Full statistical breakdown in Appendix~\ref{app:stat-sig}.

\subsubsection{Impact of Acoustic Realism}

To isolate which factors hurt performance most, we conduct ablations on the Retail domain, adding noise, accents, or user behaviors independently (Table~\ref{tab:ablation-single}).

\begin{table}[h]
\caption{Ablation: impact of individual acoustic factors on pass@1 (Retail domain).}
\label{tab:ablation-single}
\centering
\begin{small}
\resizebox{\columnwidth}{!}{%
\begin{tabular}{lccc|c}
\toprule
\textbf{Condition} & \textbf{Google} & \textbf{OpenAI} & \textbf{xAI} & \textbf{All} \\
\midrule
Clean & 39\% & 39\% & \textbf{42\%} & 40\% \\
+ Noise & \textbf{37\% (-1)} & 26\% (-13) & 29\% (-12) & 31\% (-9) \\
+ Accents & \textbf{36\% (-2)} & 21\% (-18) & 23\% (-18) & 27\% (-13) \\
+ Interrupts & \textbf{38\% (+0)} & 30\% (-8) & 36\% (-5) & 35\% (-4) \\
Realistic & \textbf{28\% (-11)} & 15\% (-23) & 20\% (-21) & 21\% (-19) \\
\bottomrule
\end{tabular}%
}
\end{small}
\end{table}

\textbf{Accents are the most damaging factor}, causing a 13pp average drop (vs 9pp for noise, 5pp for interrupts). This finding has accessibility implications: users with non-American accents may face systematically worse service. OpenAI and xAI are particularly vulnerable to accents ($-$18pp each), while Google shows greater robustness ($-$2pp). Because accents are implemented via TTS personas, these results should be interpreted as indicative rather than definitive.

\textbf{Google is consistently the most robust provider} across individual ablation conditions, with minimal degradation from noise ($-$2pp) or interrupts ($-$1pp). However, Google's robustness to isolated factors does not fully transfer to compound stress: individual effects sum to just $-$5pp, yet the full Realistic condition causes $-$11pp---suggesting super-additive interactions when multiple factors combine. This 11pp drop still compares favorably to 22--24pp for competitors.

\subsubsection{Voice Interaction Quality}

Beyond task completion, we evaluate conversational dynamics under Realistic conditions (Table~\ref{tab:voice-quality}). We report four aggregate dimensions: \textbf{Latency} (how quickly agents react), \textbf{Responsiveness} (whether agents act when needed), \textbf{Interrupt} (how often agents cut off users mid-speech), and \textbf{Selectivity} (whether agents correctly ignore signals that do not require action).

\begin{table}[h]
\caption{Voice interaction quality (Realistic condition, aggregated across domains). \textbf{Bold} indicates best. Full breakdown in Appendix~\ref{app:voice-metrics-detail}.}
\label{tab:voice-quality}
\centering
\begin{small}
\resizebox{\columnwidth}{!}{%
\begin{tabular}{@{}lcccc@{}}
\toprule
\textbf{Provider} & \textbf{Latency}$\downarrow$ & \textbf{Responsiveness}$\uparrow$ & \textbf{Interrupt}$\downarrow$ & \textbf{Selectivity}$\uparrow$ \\
\midrule
Google & 1.13s & 71\% & \textbf{24\%} & 51\% \\
OpenAI & 2.22s & 68\% & 34\% & \textbf{74\%} \\
xAI & \textbf{0.99s} & \textbf{85\%} & 104\% & 51\% \\
\bottomrule
\end{tabular}
}
\end{small}
\end{table}

\textbf{xAI achieves the best latency and responsiveness}: fastest reactions (0.99s average latency) and highest responsiveness (85\%). However, this speed comes at a severe cost: xAI has an interrupt rate of 104\%---interrupting users more than once per turn on average.

\textbf{OpenAI shows the opposite trade-off}: slowest latency (2.22s) and lowest responsiveness (68\%), but highest selectivity (74\%) and a moderate interrupt rate (34\%). OpenAI is more conservative, waiting longer to ensure genuine user intent before responding.

\textbf{Google achieves the best balance}: lowest interrupt rate (24\%), reasonable latency (1.13s), and mid-range selectivity (51\%), though with lower responsiveness (71\%). No provider achieves both high responsiveness and low interruption, highlighting the fundamental challenge of real-time turn-taking.

\subsection{Qualitative Error Analysis}
\label{sec:analysis}

To characterize failure modes beyond aggregate pass rates---and to verify that observed failures stem from agent behavior rather than artifacts of the benchmark or user simulator---we perform a qualitative error analysis.

\paragraph{Task Selection.} We define $\text{pass}_{\text{text}}$ as tasks where both GPT-4.1 and GPT-5.2 (medium reasoning) succeed in text mode, $\text{pass}_{\text{clean}}$ as tasks where a majority of audio providers succeed under Clean conditions, and $\text{pass}_{\text{realistic}}$ as tasks where a majority succeed under Realistic conditions. We construct two analysis cohorts:
\begin{itemize}[nosep,leftmargin=*]
    \item \textbf{Voice-Fragile}: Tasks that satisfy $\text{pass}_{\text{text}}$ but not $\text{pass}_{\text{clean}}$, isolating inherent voice interaction challenges.
    \item \textbf{Noise-Fragile}: Tasks that satisfy $\text{pass}_{\text{clean}}$ but not $\text{pass}_{\text{realistic}}$, isolating the impact of acoustic realism (noise, accents, interruptions).
\end{itemize}
For each cohort, we sample 20 tasks, prioritizing those exhibiting the largest performance gap between conditions. For each sampled task, we randomly select one failing provider for analysis.

\paragraph{Annotation Procedure.} Two independent raters examined each failed simulation, labeling: (1) \textit{error source}---whether the agent or user simulator caused the first critical error; and (2) \textit{error type}---one of logical, transcription, VAD/unresponsive, hallucination, or early termination. Inter-rater agreement was 92.5\% (37/40 tasks); disagreements were resolved through discussion.

\paragraph{Results.} Table~\ref{tab:combined-error-analysis} shows the distribution of error types by source for both cohorts. Full annotations are in Appendix~\ref{app:qualitative-analysis}.

\begin{table}[h]
\caption{Error analysis: distribution of error types by source. Agent errors dominate in both cohorts (75\% and 90\%).}
\label{tab:combined-error-analysis}
\centering
\begin{small}
\resizebox{0.85\columnwidth}{!}{%
\begin{tabular}{@{}llcc@{}}
\toprule
\textbf{Source} & \textbf{Error Type} & \makecell{\textbf{Voice-}\\\textbf{Fragile}} & \makecell{\textbf{Noise-}\\\textbf{Fragile}} \\
\midrule
\multirow{5}{*}{Agent} & Logical & 8 & 6 \\
& Transcription & 4 & 4 \\
& VAD & 2 & 1 \\
& Unresponsive & 1 & 7 \\
\cmidrule{2-4}
& \textit{Total} & \textit{15 (75\%)} & \textit{18 (90\%)} \\
\midrule
\multirow{4}{*}{User} & Logical & 1 & 1 \\
& Hallucination & 3 & 1 \\
& Early Term. & 1 & -- \\
\cmidrule{2-4}
& \textit{Total} & \textit{5 (25\%)} & \textit{2 (10\%)} \\
\bottomrule
\end{tabular}
}
\end{small}
\end{table}

\textbf{Agent errors dominate}: 75\% of failures in the Voice-Fragile cohort and 90\% in the Noise-Fragile cohort are attributed to the agent rather than the user simulator---suggesting that observed failures primarily reflect agent behavior under our evaluation setup, not simulator artifacts.

\textbf{Logical errors are most common in the Voice-Fragile cohort} (8/20), indicating that voice agents struggle with reasoning even when transcription is accurate. However, \textbf{VAD/unresponsive errors become dominant in the Noise-Fragile cohort} (8/20), where background noise and interruptions cause agents to miss user utterances or become unresponsive.
